% Latex Project Proposal
% SENG 474

\documentclass[journal]{IEEEtran}

%-------------------------------------------------------------------------------------------------------------------

\begin{document}

\title{Project Proposal\\Music Genre Based on Lyrics}

\author{Anna~Sollazzo,
        Henry~Hart,
        Jordan~Jay,
        Juan~Carlos~Gallegos,
        And~Robert~Craig \\ January 31, 2018}
        
\maketitle

%-------------------------------------------------------------------------------------------------------------------

\begin{abstract}

We aim to train a classifier to predict a song's genre based on its lyrics, with the goal of exploring the relationships between the lyrical content and different genres. In this domain, we will test our intuition about each genre's distinctive attributes and investigate more opaque or subtle patterns. The model will be trained and evaluated on a dataset of 380,000 song lyrics and their genres, acquired from Kaggle.com; the genres span: country, electronic, folk, hip-hop, indie, jazz, metal, pop, R/B, and rock. We will generate features describing the lyrical content with the support of existing Natural Language Processing techniques and resources, such as the Natural Language Toolkit Python library. Attributes that we are consider evaluating include: line count; syllable count; rhyme frequency; and amount of variation in rhyme scheme. Our primary goal is to produce a feature set for the song lyrics dataset along with a single classification model, on which we will iterate to increase accuracy. Our stretch goal is to develop a second model to compare to our first model, providing insight into the tradeoffs in model selection for this particular problem. The development of the second model will depend on the opportunity for insight that we perceive in our first model.


\end{abstract}

%-------------------------------------------------------------------------------------------------------------------

\section{Related Work}

\IEEEPARstart{W}{ho} has done something like what you plan to do?  What did they try?  How well did it work? This is an important section, so don't skimp here. Search Google Scholar to find related work. \\\\https://arxiv.org/pdf/1707.04678.pdf \\ \\ https://nlp.stanford.edu/courses/cs224n/2006/fp/sadovsky\-x1n9\-1\-224n\_final\_report.pdf \\ \\ http://worldcomp-proceedings.com/proc/p2016/DMI8052.pdf\\

%-------------------------------------------------------------------------------------------------------------------

\section{Data Description and Source}

\IEEEPARstart{T}{his} is very important! If you are not working with a pre-existing dataset, you must convince me that the data for your project exists and is easily accessible (e.g. can easily be scraped from a public website) Tell me what your features (attributes, columns of your data matrix X) will be.

%-------------------------------------------------------------------------------------------------------------------

\section{Proposed project}

Data, Algorithm, Evaluation... bunch of random algorithms...

%-------------------------------------------------------------------------------------------------------------------

\newpage

\section{Estimated Timeline}

\begin{table}[h!]
  \begin{center}
    \label{tab:table1}
    \begin{tabular}{l|l|l|}
      \textbf{Due Date} & \textbf{Activity}\\

      \hline
      Feb. 6th & Project Proposal\\
      Feb. 13th & Data Cleaning and Mining Preparation\\
      Feb. 23th & Algorithm Environment Setup\\
      Feb. 27th & Run the Commonly Used Algorithms and Record Results\\
      Mar. 6th &Deep Learning Method and Record Results\\
      Mar. 13th & Midterm Report\\
      Mar. 20th & Gather Results and Prepare Final Report\\
      Mar. 28th & Project Presentation\\
      Apr. 6th & Final Report\\

    \end{tabular}
  \end{center}
\end{table}

%-------------------------------------------------------------------------------------------------------------------

\section{Distribution of Tasks Among Team Members}

Each team member is a lead in a specific area and is required to delegate tasks in their area accordingly. This will aide in equally distributing the workload and involving team members to contribute in every aspect.\\ 

The team leaders are as follows:

\begin{table}[h!]
    \label{tab:table1}
    \begin{tabular}{l|l|l|}
      \textbf{Member} & \textbf{Task}\\

      \hline
	Anna Sollazzo & TBD\\
	Henry Hart & TBD\\
	Jordan Jay & TBD\\
	Juan Carlos Gallegos & TBD\\
	Robert Craig & TBD\\

    \end{tabular}
\end{table}
 
\newpage

%-------------------------------------------------------------------------------------------------------------------

\begin{thebibliography}{1}

\bibitem{IEEEhowto:kopka}
H.~Kopka and P.~W. Daly, \emph{A Guide to \LaTeX}, 3rd~ed.\hskip 1em plus
  0.5em minus 0.4em\relax Harlow, England: Addison-Wesley, 1999.

\end{thebibliography}

%-------------------------------------------------------------------------------------------------------------------

\end{document}
