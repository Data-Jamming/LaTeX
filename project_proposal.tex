% Latex Project Proposal
% SENG 474

\documentclass[journal]{IEEEtran}

%-------------------------------------------------------------------------------------------------------------------

\begin{document}

\title{Project Proposal:\\\emph{Music Genre Classification Based on Song Lyrics}}

\author{Anna~Sollazzo,
        Henry~Hart,
        Jordan~Jay,
        Juan~Carlos~Gallegos,
        And~Robert~Craig \\ January 31, 2018}
        
\maketitle

%-------------------------------------------------------------------------------------------------------------------

\begin{abstract}

    We aim to explore the relationships between lyrical content and different music genres by constructing multiple classifiers to predict a song's genre based solely on its lyrics. In this domain, we will test our intuition about each genre's distinctive attributes and investigate more opaque and subtle patterns. We intend to implement a traditional classifier trained on a set of Natural Language Processing (NLP) features derived from the lyrics as well as a deep learning model; we will then compare and contrast their respective accuracies and areas of error.\par

\end{abstract}

%-------------------------------------------------------------------------------------------------------------------
\section{Introduction}

    \IEEEPARstart{O}{ur} models will be trained and tested on a dataset that contains 380,000 song lyrics and their corresponding genres, available on Kaggle.com and originally scraped from metrolyrics.com \cite{KaggleDataset}; the dataset is comprised of a song title, year, artist, genre, and lyrics and it spans 10 genres: country, electronic, folk, hip-hop, indie, jazz, metal, pop, R\&B, and rock. Before training, we will preprocess the data by cleaning it (e.g. removing the relatively few songs without given genres) and generating NLP features to describe the data; features such as part-of-speech (POS) frequency, line/word/syllable count, and rhyme scheme will be evaluated using NLP techniques and tools such as Python NLTK \cite{NLTK}.\par

We intend to use 70\% of our data for training, 10\% for validation, and 20\% for testing. While the neural network will be given the raw, unstructured lyric data, the SVM will be run on a linguistic feature set derived from the lyrics. This feature set will echo the themes seen in the relevant literature, taking into count elements such as POS frequency, word count, syllable count, and rhyme scheme. Once both of our models are completed, we will evaluate their classification precision and accuracy. Furthermore, we intend to analyse the dispersal of model errors over classes to check our intuition regarding both models. We would like to understand their areas of confusion: for example, whether the models frequently misclassify a song of a particular genre as another.


%-------------------------------------------------------------------------------------------------------------------

\section{Related Work}

\IEEEPARstart{T}{here} exist numerous attempts at classifying songs into genres solely by their lyrical content using neural networks and more traditional models such as SVM, k-nearest neighbor, etc.
Tsaptsinos \cite{tsaptsinos} extended the use of hierarchical attention networks (HAN) -- a type of recurrent neural network previously proposed and employed by Yang et al. \cite{Yang} for document classification -- to the lyrics-based music genre classification problem, demonstrating the model's ability to outperform the Long Short-Term Memory (LSTM) model when classifying across 117 genres. The HAN model performed particularly well when attention was placed at the line-level, as opposed to the segment level; that is, the model's accuracy benefited from considering lines individually instead of a block of lines (e.g. a refrain) as a whole unit. \par

More traditional models such as SVM, k- nearest neighbor, and random forests have found comparable success in lyrics-based genre classification. However, these models required preprocessing of the lyrics using NLP techniques to generate lyric feature sets for training. Canicatti \cite{canicatti} used song metadata (length and beats per minute), as well as word frequency vectors derived after stop word removal and word stemming, to train four models to classify five genres. Random forests outperformed the others, with a top accuracy of 47.35 percent. \par

Similarly, Mayer et al. \cite{mayer} trained several classifiers on basic features (e.g word count, line count) as well as more complex statistics such as Part-of-Speech (POS) tag frequency, occurrences of simple rhyme schemes (e.g. AABB, ABAB, ABBA and AA), and the proportion of unique terms used in rhyming. Their SVM that was trained on word count, line count, POS data, and rhyme data performed best, with an accuracy of 33.47 percent over ten genres. The team concluded that, in all cases, the inclusion of derived features in training data outperformed the traditional bag of words approach. Notably, both Canicatti and Mayer et al. suggested that a portion of their error may have come from their choice to remove lyrical annotation denoting repetition (e.g. "[chorus]" or "(x2)" ) instead of expanding them with the appriopriate sections (e.g. lyrics of chorus).\par

%-------------------------------------------------------------------------------------------------------------------

\section{Distribution of Tasks Among Team Members}

Each team member is a lead in a specific area and is required to delegate tasks in their area accordingly. This will aid in equally distributing the workload and involving team members to contribute in every aspect.\\ 

The team leaders are as follows:

\begin{table}[h!]
    \label{tab:table1}
    \begin{tabular}{l|l|l|}
      \textbf{Member} & \textbf{Task}\\

      \hline
	Anna Sollazzo & Natural Language Feature Set/SVM\\
	Henry Hart & Data Preprocessing/SVM\\
	Jordan Jay & Data Preprocessing/SVM\\
	Juan Carlos Gallegos & Neural Net\\
	Robert Craig & Neural Net\\

    \end{tabular}
\end{table}

\newpage

%-------------------------------------------------------------------------------------------------------------------

\section{Estimated Timeline}

\begin{table}[h!]
  \begin{center}
    \label{tab:table1}
    \begin{tabular}{l|l|l|}
      \textbf{Due Date} & \textbf{Activity}\\

      \hline
����� 	Feb.	6th & Project Proposal\\
	Feb. 13th & Finish Data Cleaning and Mining Preparation\\
	& Start Both Methods; SVM and Recurrent Neural Net\\
 �����	Feb. 	23th & Progress on 'Rock or Not' (SVM) Recorded\\
	& Progress on Recurrent Neural Net Recorded\\
 �����	Feb. 	27th & Finish 'Rock or Not' Further the Analysis to 'Rock or Pop' etc.\\
	& Progress on the Neural Net Method Recorded\\
 ����	Mar. 	6th & Finish SVM and Record Results, Start Midterm Report\\
	& All Hands On Neural Net\\
 ���	Mar. 	11th & Prepare and Finalize Midterm Report\\
	& All Hands On Neural Net\\
 �����	Mar. 	13th & Midterm Report Due \\
	& Continue Neural Net\\
 �����	Mar. 	20th & Gather Results of Both Methods and Prepare Final Report\\
 �����	Mar. 	25th & Meeting to Organize Presentation\\
	& Finish Final Report\\
 �����	Mar. 	28th & Project Presentation\\
 �����	Apr. 	6th & Final Report Due\\

    \end{tabular}
  \end{center}
\end{table}

%-------------------------------------------------------------------------------------------------------------------

\begin{thebibliography}{6}

\bibitem{tsaptsinos}
A.~Tsaptsinos, \emph{Lyrics-Based Music Genre Classification Using a Hierarchical Attention Network}, \relax ICME, Stanford University, USA, 2017.

\bibitem{NLTK}
Bird, Steven, Edward Loper and Ewan Klein (2009), \emph{Natural Language Processing with Python}. \relax O'Reilly Media Inc.

\bibitem{Yang}
Z. Yang, D. Yang, C. Dyer, X. He, A. Smola, E. Hovy (2014), \emph{Hierarchical Attention Networks for Document Classification}.

\bibitem{KaggleDataset}
GyanendraMishra (2017), \emph{380,000+ Lyrics from MetroLyrics}, https://www.kaggle.com/gyani95/380000-lyrics-from-metrolyrics.

\bibitem{canicatti}
A.~Canicatti (2016), \emph{ "Song Genre Classification via Lyric Text Mining."} \relax Proceedings of the International Conference on Data Mining (DMIN). The Steering Committee of The World Congress in Computer Science, Computer Engineering and Applied Computing (WorldComp).

\bibitem{mayer} (2008) R.~Mayer, R.~Neumayer, and A.~Rauber., \emph{" Rhyme and Style Features for Musical Genre Classification by Song Lyrics"} \relax Vortrag: International Conference on Music Information Retrieval (ISMIR) Philadeliphia, USA; 14.09. 2008-18.09. 2008; in:" Proceedings of the 9th International Conference on Music Information Retrieval",(2008), S. 337-342.

\bibitem{kaggle} Kaggle~Datasets (2017), \emph{380,000+ Lyrics From MetroLyrics}, \relax https://www.kaggle.com/gyani95/380000-lyrics-from-metrolyrics

\end{thebibliography}

%-------------------------------------------------------------------------------------------------------------------

\end{document}
